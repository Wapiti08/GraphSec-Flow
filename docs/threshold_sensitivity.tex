\subsubsection{Threshold Sensitivity (Lead Detection)}

Threshold sensitivity is evaluated on a same-scale subgraph (i.e., a subgraph with the same number of nodes as a full evaluation window),
ensuring controlled comparison across thresholds. This subgraph preserves the global scale characteristics of the full dependency graph, making the resulting threshold choice stable and transferable to the full-graph evaluation setup.

\begin{table}[t]
\centering
\caption{Sensitivity of Lead Detection Threshold.}
\label{tab:lead-threshold}
\begin{tabular}{lccccc}
\toprule
\textbf{Thres.} &
\textbf{MRR $\uparrow$} &
\textbf{Hit$_{comm}$ $\uparrow$} &
\textbf{Cove$_{comm}$ $\uparrow$} &
\textbf{Latency(ms) $\downarrow$} \\
\midrule
0.3 & 0.00111 & 0.025 & 0.025 & 0.45 \\
0.5 & 0.00129 & 0.025 & 0.025 & 0.44 \\
0.8 & 0.00208 & 0.025 & 0.025 & 0.49 \\
1.0 & 0.00280 & 0.025 & 0.025 & 2.61 \\
\bottomrule
\end{tabular}
\end{table}

To determine an appropriate lead detection threshold for large-scale evaluation, we first conduct a sensitivity analysis using ground-truth paths constructed with a dependency expansion depth of \textbf{10}.
This deeper context ensures that potential long-range vulnerability propagation are fully covered and provides a reliable basis for threshold calibration.

As shown in Figure~\ref{tab:lead-threshold}, varying the lead detection threshold from $0.3$ to $1.0$ revals a consistent trade-off between ranking precision and stability.  
Lower thresholds (e.g., $th{=}0.3$–$0.5$) yield noisier activations with little improvement in accuracy, while overly high thresholds (e.g., $th{=}1.0$) increase computational overhead without significant gain in coverage. A balanced setting of \textbf{$th{=}0.8$} achieves stable and reliable ranking performance (MRR~$\approx$~0.002) 
and is therefore adopted as the default threshold for all subsequent experiments.

