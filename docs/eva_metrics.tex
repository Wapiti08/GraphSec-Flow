\subsubsection{Ranking Performance} 
\begin{enumerate}
    \item \textbf{Mean Reciprocal Rank (MRR)} evaluates how well the model elevates true vulnerable targets under a global ranking:
    \begin{align}
        \mathrm{MRR} = \frac{1}{N} \sum_{i=1}^{N} \frac{1}{r_i},
    \end{align}
    where $r_i$ denotes the first correct rank.
    It captures overall ranking sensitivity across long-tailed distributions. 
\end{enumerate}


\subsubsection{Temporal Responsiveness}
\begin{enumerate}
    \item \textbf{LeadTime} measures how early the model can flag a target before its first confirmed disclosure:
    \begin{align}
    \mathrm{LeadTime}=\frac{1}{N} \sum_{i=1}^{N} (t_{\text{disclose},i} - t_{\text{alert},i})
    \end{align}
    where $t_{\text{alert},i}$ and $t_{\text{disclose,i}}$ denote the alert and disclosure timestamps, respectively.  
    Positive values indicate successful proactive detection prior to external disclosure.
    LeadTime serves as a temporal robustness measure under realistic CVE release delays.
\end{enumerate}


\subsubsection{Root-Cause Localization Metrics}
To evaluate if the model can pinpoint the origin of a vulnerability cascade, we use:
\begin{enumerate}
    \item \textbf{Hit-Root}
    Binary success indicator:
    
    \begin{align}
    \mathrm{Hit\text{-}Root} = \mathbf{1}\bigl(\mathcal{R}_{\text{pred}} \cap \mathcal{R}_{\text{gt}} \neq \emptyset\bigr)
    \end{align}
    where $R_{pred}$ and $R_{gt}$ denote predicted and ground-truth roots.

    \item \textbf{RootCoverage}

    \begin{align}
    \mathrm{RootCoverage} = \frac{|\mathcal{R}_{\text{pred}} \cap \mathcal{R}_{\text{gt}}|}{|\mathcal{V}_{\text{gt}}|},
    \end{align}
    
    Fraction of affected nodes encompassed by the inferred root region.

    \item \textbf{RootPurity}
    
    \begin{align}
    \mathrm{RootPurity} = \frac{|\mathcal{R}_{\text{pred}} \cap \mathcal{R}_{\text{gt}}|}{|\mathcal{R}_{\text{pred}}|},
    \end{align}
    
    Precision of the predicted root community.

    \item \textbf{RootRankInComm}
    Mean rank of the ground-truth root within its structural community, capturing intra-community localization sensitivity.
    
\end{enumerate}


\subsubsection{Propagation Path Identification}

We assess alignment between predicted and reference propagation chains:
\begin{enumerate}
    \item \textbf{Path-F1 (node-overlap)}
    \begin{align}
    \mathrm{Path\text{-}F1}
    = \frac{2 \times |\hat{V} \cap V^*|}{|\hat{V}| + |V^*|},
    \end{align}
    where $\hat{V}$ and $V^*$ are predicted and ground-truth node sets.
    We report both a Jaccard variant \cite{niwattanakul2013using} and a GT-recall variant, capturing strict vs. tolerant overlap criteria.

    \item \textbf{PredictedPathCount}
    Number of hypothesized propagation chains inferred by the model;
    indicates how conservative or expansive the model’s propagation exploration is.

\end{enumerate}


\subsubsection{Community-Aware Performance}

To assess the benefits of structural grouping, we evaluate two community-level metrics.  
\begin{enumerate}
    \item \textbf{community hit rate}
    \begin{align}
    \mathrm{Hit_{comm}}
    =
    \frac{\text{Correct community hits}}
         {\text{Total vulnerable nodes}}.
    \end{align}
    measures whether vulnerability predictions land in the correct structural region.  
    \item \textbf{community coverage} of the predicted root region is
    \begin{align}
    \mathrm{CommCoverage}
    =
    \frac{|\mathcal{C}_{\mathrm{root}}\cap \mathcal{T}|}{|\mathcal{T}|},
    \end{align}
    where $\mathcal{C}_{\mathrm{root}}$ is the inferred root community 
    and $\mathcal{T}$ is the set of ground-truth affected nodes.   
    This provides a continuous notion of community-level recall and complements 
    $\mathrm{Hit_{comm}}$ by quantifying how well the selected community 
    captures the true spread of vulnerability impact.
\end{enumerate}

