\subsection{Influence of Graph Expansion Depth}
Evaluating structural models on the full dependency graph is challenging because the ecosystem-scale network is extremely sparse and dominated by semantically unrelated nodes.  
Such sparsity dilutes vulnerability signals and obscures propagation structure, making it difficult to obtain meaningful measurements of ranking, localization, or path alignment.  

To enable fair and diagnostically useful evaluation, we therefore analyze how different neighborhood expansion depths affect both the density of actionable information and the computational overhead.
Specifically, we compare three structural settings for subgraph construction:
\begin{itemize}
    \item 2-hop expansion: includes the dependency neighborhood reachable within two hops from each ground-truth path.
    \item 3-hop expansion: further expands each node’s local vicinity, producing an order-of-magnitude larger subgraph.
    \item Full graph: uses the entire dependency network without truncation.
\end{itemize}
Empirically, the choice of hop depth strongly influences information density.
A 2-hop expansion yields a compact yet semantically cohesive subgraph
(58K nodes, 481K edges), fully retaining all ground-truth nodes (16/16).
Expanding to 3 hops increases the graph size by 12.7× (740K nodes) and introduces substantial noise—the proportion of CVE-associated nodes drops from 2.10\% (2-hop) to 1.27\% (3-hop).
The full graph further dilutes actionable signal (0.48\% CVE density) and exhibits significantly higher latency.

Across all evaluated models, the 2-hop setting consistently yields the strongest prediction signal,
often outperforming both 3-hop and full-graph settings by one to two orders of magnitude in MRR and RootPurity.
In contrast, 3-hop expansions broaden the graph but degrade ranking quality,
suggesting that excessive structural expansion injects far more noise than useful context.

Based on this analysis, all subsequent ranking experiments use the 2-hop-expanded graph,
which achieves the best trade-off between predictive accuracy, computational cost,
and preservation of ground-truth propagation structure.
